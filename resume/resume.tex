%%%%%%%%%%%%%%%%%
% This is an sample CV template created using altacv.cls
% (v1.1.2, 1 February 2017) written by LianTze Lim (liantze@gmail.com). Now compiles with pdfLaTeX, XeLaTeX and LuaLaTeX.
%
%% It may be distributed and/or modified under the
%% conditions of the LaTeX Project Public License, either version 1.3
%% of this license or (at your option) any later version.
%% The latest version of this license is in
%%    http://www.latex-project.org/lppl.txt
%% and version 1.3 or later is part of all distributions of LaTeX
%% version 2003/12/01 or later.
%%%%%%%%%%%%%%%%

%% If you need to pass whatever options to xcolor
\PassOptionsToPackage{dvipsnames}{xcolor}

%% If you are using \orcid or academicons
%% icons, make sure you have the academicons
%% option here, and compile with XeLaTeX
%% or LuaLaTeX.
% \documentclass[10pt,a4paper,academicons]{altacv}

%% Use the "normalphoto" option if you want a normal photo instead of cropped to a circle
% \documentclass[10pt,a4paper,normalphoto]{altacv}

\documentclass[10pt,letter]{altacv}

%% AltaCV uses the fontawesome and academicon fonts
%% and packages.
%% See texdoc.net/pkg/fontawecome and http://texdoc.net/pkg/academicons for full list of symbols.
%%
%% Compile with LuaLaTeX for best results. If you
%% want to use XeLaTeX, you may need to install
%% Academicons.ttf in your operating system's font
%% folder.


% Change the page layout if you need to
\geometry{left=1cm,right=9cm,marginparwidth=7.25cm,marginparsep=0.75cm,top=0.5cm,bottom=1cm}

% Change the font if you want to.

\usepackage[none]{hyphenat}
\usepackage{setspace}
\usepackage[document]{ragged2e}

\setmainfont{avenir}


% Palette :: https://coolors.co/fffffa-22223b-1d7874

\definecolor{SpaceCadet}{HTML}{22223B}
\definecolor{Skobleoff}{HTML}{1D7874}
\definecolor{BabyPowder}{HTML}{FFFFFA}

%

\colorlet{heading}{Skobleoff}
\colorlet{accent}{Skobleoff}
\colorlet{emphasis}{SpaceCadet}
\colorlet{body}{SpaceCadet}

\pagecolor{BabyPowder}

\hypersetup{
  colorlinks=false,
  urlbordercolor=BabyPowder
}

% Change the bullets for itemize and rating marker
% for \cvskill if you want to
\renewcommand{\itemmarker}{{\small\textbullet}}
\renewcommand{\ratingmarker}{\faCircle}

\begin{document}
\name{Tyler Holewinski}
\tagline{Software Engineer}
\personalinfo{%
  % Not all of these are required!
  % You can add your own with \printinfo{symbol}{detail}
  \email{tyler@holewinski.dev}\\\smallskip
  \phone{(719) 822-5878}\\\smallskip
  \homepage{www.holewinski.dev}\\\smallskip
  \github{github.com/erwijet}\\\smallskip
  \linkedin{in/tylerholewinski}\\\smallskip
  %% You MUST add the academicons option to \documentclass, then compile with LuaLaTeX or XeLaTeX, if you want to use \orcid or other academicons commands.
%   \orcid{orcid.org/0000-0000-0000-0000}
}

%% Make the header extend all the way to the right, if you want.
\begin{fullwidth}
\marginpar{\makesidebarheader\vspace{1.5pt}\smallskip

\cvsection{Education}
\cvevent{Rochester Institute of Technology}{B.S. Individualized Studies\\Software Development \& Mathematics}{2021 --- 2024}{}

\medskip

\cvsection{Skills}
\cvsubsection{Languages}
\cvtag{TypeScript}
\cvtag{JavaScript}
\cvtag{Rust}
\cvtag{C}
\cvtag{HTML/CSS}
\cvtag{SCSS/SASS}
\cvtag{Python}
\cvtag{Elixir}
\cvtag{Swift}
\cvtag{Svelte}
\cvtag{Kotlin}

\medskip

\cvsubsection{Tools}
\cvtag{React}
\cvtag{Express.js}
\cvtag{Vite}
\cvtag{Tailwind}
\cvtag{Webpack}
\cvtag{NGINX}
\cvtag{Springboot}
\cvtag{Git}
\cvtag{GitHub}
\cvtag{SwiftUI}
\cvtag{Bash}
\cvtag{SQL}
\cvtag{Docker/OCI}
\cvtag{WebAssembly}
\cvtag{MongoDB}

%% Yeah I didn't spend too much time making all the
%% spacing consistent... sorry. Use \smallskip, \medskip,
%% \bigskip, \vpsace etc to make ajustments.
\medskip
\cvsection{Activities}

\cvevent{Computer Science House}{Member}{Spring 2021 - Present}{}
CSH is a student-run organization at RIT providing a productive, inclusive, and educational community for anyone interested in Computer Science.

\medskip

\cvsubevent{Speaker}{Spring 2021 - Present}{}
Gave talks to CSH and the RIT community on Elixir/Erlang, Vim/Neovim, and VSCode.

\medskip

\cvevent{Club Cafe}{Member}{Spring 2021}{}
Club Cafe is an organization for members of the RIT community to meet, share brewing techniques, and discuss third-wave specialty coffee.}
    \vspace*{-1\baselineskip}
\makecvheader
\end{fullwidth}
%% Provide the file name containing the sidebar contents as an optional parameter to \cvsection.
%% You can always just use \marginpar{...} if you do
%% not need to align the top of the contents to any
%% \cvsection title in the "main" bar.

%% Make the columns line up
\vspace{14pt}

\cvsection{Experience}
\cvevent{\href{https://bryx.com}{Bryx}}{Software Engineer}{May 2022 -- Present}{Rochester, NY}
\begin{itemize}
  \item Building a cloud-based records management solution for first responders
  \item Driving design and implementation for an in-house React UI library in Tailwind
  \item Developing a real-time validation engine in React with a Rust-based DSL for rule expression
\end{itemize}
\textit{\textbf{Tools:} React, Typescript, Rust, Kotlin, Tailwind, WebAssembly}

\divider

\cvevent{\href{https://www.rit.edu/its/}{RIT ITS}}{Helpdesk Technician}{September 2021 -- May 2022}{Rochester, NY}
\begin{itemize}
  \item Worked with multiple internal departments of RIT to provide a seamless a customer service experience
  \item Helped users with IT support such as RIT LDAP and GSuite account configuration and repair
  \item Assisted with development of internal support tools
\end{itemize}
\textit{\textbf{Tools:} Javascript, Peoplesoft, LDAP, Userscript}

\smallskip

\cvsection{Projects}
\project{Aspen}{https://github.com/erwijet/aspen}
\begin{itemize}
\item Allowed users to leverage browser site search to quickly jump to frequent sites
\item Used keyword-based indexing, as well as fuzzy search for link resolution
\item Provided a web and iOS client that leverage gRPC for cross-framework, real-time updates
\end{itemize}
\textit{\textbf{Tools:} Rust, Tonic, React, Typescript, gRPC, Swift, SwiftUI, Python, FastAPI, MongoDB}

\divider

\project{Bagelbot}{https://github.com/bagelbotdev/api}
\begin{itemize}
\item Provided a centralized way for Slack users to open and close public tabs for \href{https://www.balsambagels.com}{Balsam Bagels}
\item Maintained a credit system to purchase bagels replenished by hosting a tab
\item Reverse-engineered the \href{https://www.balsambagels.com}{Balsam Bagel} ToastTab API to build carts from user orders
\end{itemize}
\textit{\textbf{Tools:} Typescript, Express.js, MongoDB, Chakra, Kubernetes}

\divider

\project{Tigerwatch}{https://github.com/erwijet/tigerwatch}
\begin{itemize}
\item Reverse-engineered API that pulls RIT dining system transaction history and balances across multiple meal plan accounts
\item Enables students to more easily view information about their meal plan spenddown
\item Provides extensive documentation to the authentication and data flow for future Computer Science House members to build their own appsi
\end{itemize}
\textit{\textbf{Tools:} Typescript, React Elixir, Cowboy/Plug, Rust, Rocket}

\clearpage

\end{document}
